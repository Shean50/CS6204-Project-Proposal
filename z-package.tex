

% ---------------------------------------------------------------
% IMPORTANT NOTES: Dear students, please go to the end of this file, and see
% an important note.
% ---------------------------------------------------------------

%\usepackage{oneinchmargins,doublespace,times}
\usepackage{S/oneinchmargins}  % in ~/Library/texmf/tex/latex/base/ or in S/
\usepackage{times}
\usepackage{relsize}
\usepackage{enumerate}
\usepackage{graphicx}
\usepackage{url}  % -- need \sb -- so never redifined \sb
\usepackage[usenames,dvipsnames]{xcolor}
\usepackage[bookmarks=false]{hyperref}

\hypersetup{colorlinks=true,
citecolor=Maroon,
linkcolor=Green,
urlcolor=Maroon}

% \usepackage{threeparttop}

% \include{mathpmtc} % math stuff same font as main tex

\usepackage{breakurl}  % -- need \sb -- so never redifined \sb
\usepackage{setspace}
\usepackage{rotating}
\usepackage{xspace}

\usepackage{floatflt}
\usepackage{wrapfig}
\usepackage{alltt}
\usepackage{epstopdf}
\usepackage{subfigure}

%\usepackage{listings}
%\usepackage{algorithm}
%\usepackage{algorithmic}

\usepackage{fancyvrb}
%\usepackage{ulem} % for strike out,  
% \em are now underline and and \sout are now strikes, use \it for italic
% BUT never do this because \em in the references conferences
% will become underline!!
\renewcommand{\ttdefault}{cmtt}




% make sure url bib break point does not
% give undefull hbox message and the break line 
% is really nice now
\usepackage{etoolbox}
\apptocmd{\thebibliography}{\raggedright}{}{}

% \usepackage{natbib}  % don't sort  
\usepackage[sort&compress]{natbib} % sort the bib

\usepackage{colortbl} % for table color
%\usepackage{pstricks} % for gray hline
%\input{colortab} % for gray hline (must include pstricks)
%\usepackage{array}


\usepackage{wasysym}  % for whitecircle special $\ocircle$
\usepackage{pifont}   % for blackcircle \ding


\usepackage{upgreek}
\usepackage{amssymb} % for check mark, tick mark , Box, Diamond


\usepackage{tikz, pgfplots, pgfplotstable}
\usetikzlibrary{patterns}
\usepgfplotslibrary{fillbetween}
\usepackage{upgreek}




% -------------------------------------------------
% INVERTING COLOR:  search for ``UNCOMMENT''
% -------------------------------------------------
% \if 0
\newtoggle{invertcolor} % declare new toggle

                          % ------------------
% \toggletrue{invertcolor}  % COMMENT OUT THIS LINE for NORMAL COLOR
                          % see toggle.tex 
                          % ------------------

% \toggletrue{invertcolor}  % COMMENT OUT THIS LINE for NORMAL COLOR




\definecolor{mypapercolor}{HTML}{FFDDBB}  % define my brownish paper color

% this is the definition for inverting all global color including 
% text, table, and figure color (it will be ``white'' if invertcolor=true)

% SOMEHOW the text below depends on 
% \usepgfplotslibrary{fillbetween} to be enabled above
\def \invertcolorglobal {
\iftoggle{invertcolor}{
  \makeatletter
  \makeatother
  \AtBeginDocument{
    \color{white}\global\let\default@color\current@color
  }
  \color{mypapercolor}
}{
% nothing
}}


% this is the macro for invertcolortext for the body of the main paper
% (it will be ``brownish'' if invertcolor=true)
\def \invertcolortext {
\iftoggle{invertcolor}{
  \color{mypapercolor}
}{
% nothing
}}

% now before we enter begin document
% let's declare that we want to do invertcolor global
\invertcolorglobal



% \fi


% -----------------------------------------------------

% IMPORTANT NOTES REGARDING BROKEN PACKAGES

% If you 'make' fails, i.e. you cannot compile the latex, it means
% your computer doesn't have the packages.  So this is different
% options that you can do:

% 1) Comment out the package that is broken and add a comment after
% that stating who you are (e.g.  % \usepackage{breakurl} % fails in
% yourNameHere's computer).  And see if the make progresses or not. If
% you don't see the error again, that's good, perhaps we can skip the
% package for now.  If you still see a related error, it means in this
% paper we need that package and you need to install that latex
% package(s) to your computer.

% 2) Install the latex package(s).  For UC students, some instructions
% are available in the wiki.  For Score students, I've sent an email
% to the group about this.  If still confused, talk to other students
% in the group.

% ---------------------------------------------------------------




